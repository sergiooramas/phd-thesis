%!TEX root = ../thesis_a4.tex
\addtocontents{toc}{\protect\addvspace{2.25em plus 1pt}}
\bookmarksetup{startatroot}

\chapter{Summary and future perspectives}
\label{sec:conclusion}

\section{Introduction}

When this thesis started, there was almost no published work related with the extraction of high level semantic representations from music unstructured texts, although, in the context of MIR, a growing number of research works using text information \cite{}, and already structured semantic information \cite{} were available. The first knowledge extraction approaches \cite{Tata2010,Knees2011,Sordo2012} given some insights on the epistemic potential of text for music applications. In this thesis we have followed this ideas, deepening in the linguistic processing applied to extract the information, and proposing new approaches that exploit the extracted information in MIR tasks such as music recommendation and classification. In addition, we have combined extracted semantic information with content from other data modalities such as audio and images using deep neural networks. New data representations learned from the different modalities and their combination have shown to outperform hand-crafted audio features and single modality approaches.

We started with an introduction to natural language processing and representation learning in the context of MIR. In addition, we introduced the music recommendation and classification tasks (Chapter~\ref{sec:introduction}). We continued by illustrating some background concepts related to natural language understanding and summarizing the existing literature on text-based approaches in the context of MIR and recommender systems. Then, we described a framework for entity linking and the creation of a large corpus of annotated musical entities (Chapter~\ref{sec:linking}. We next proposed a method for extracting semantic relations between musical entities from unstructured texts, we evaluate the suitability of extracted knowledge to provide explanations of music recommendations (Chapter~\ref{sec:kb}). Two experiments on the applications of knowledge extraction for musicological studies are exposed (Chapter~\ref{sec:musicology}). Then, we presented knowledge-based approaches for artist similarity, music classification (Chapter\ref{sec:similarity}, and music recommendation (Chapter~\ref{sec:kb-rec}). Finally, we presented a multimodal deep learning approach for cold-start music recommendation (Chapter~\ref{sec:cold-rec}) and multi-label genre classification (Chapter~\ref{sec:multimodal-class}.

In each chapter, we provided a summary of the conclusions and relevant results of the corresponding work. In what follows, we enumerate the main contribution of this thesis~\ref{sec:conclusion:summary}. Finally, we end this dissertation with a discussion about future research directions~\ref{sec:conclusion:future}.

\section{Summary of contributions}
\label{sec:conclusion:summary}


\section{Directions for future research}
\label{sec:conclusion:future}

