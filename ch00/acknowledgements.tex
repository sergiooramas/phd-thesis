I can say now, after four and a half years working on this thesis, that so far these have been the greatest and most challenging years of my life. I learned a lot of things during these years, and surely not only about \emph{tags}. There are many people I want to thank for having contributed, in one way or another, to make this happen. First of all, I would like to thank Xavier Serra, not only for giving me the opportunity to join the MTG and supervising this thesis, but also for having lead the MTG for more than 20 years and always being enthusiastic about new projects and ideas. I remember, when I was about to apply for the grant that then supported my research, Xavier told me that pursuing a PhD is a ``way of life''. He was right, and has helped me since then. However, there is someone else to whom I feel much obliged for helping me out. After I did my first presentation on what my thesis was going to be about, Joan Serrà  told me that he was interested and, if I wanted, he could help me. I was not aware at the time of how important that collaboration would become, nor about how much I would learn from him. Fortunately I knew enough to say \emph{yes}, and so Joan became co-supervisor of the thesis. Since then, his help at all stages has been invaluable.

A very important element of this thesis has been Freesound. Working with Freesound has been a great motivation throughout the thesis, and it has turned me into a developer that now reads programming books and Python blogs. All because I have been lucky enough to be part of the Freesound team, and to work with people like Gerard Roma, Alastair Porter, Bram de Jong, and former Freesound team members Vincent Akkermans, Stelios Togias and Jordi Funollet. To all of you, I sincerely thank you for teaching me so many \emph{geeky} things, research and programming philosophy. Also related to Freesound, I want to particularly thank the Freesound moderators and all Freesound users that participated in my online experiments and that contribute everyday to make Freesound such an amazing site.

There is many other people at the MTG without whom this journey would not have been half as enjoyable. My lunch mates from the ``Lunch time conversations (official thread)'' Skype group: Panos Papiotis, Sergio Oramas, Sebastian Mealla, Oriol Romaní, Giuseppe Bandeira, Dara Dabiri, Álvaro Sarasúa, Juanjo Bosch, Martí Umbert, Carles F. Julià, and everyone else with whom I shared thoughts on the thesis or about any other professional or personal aspects: Sankalp Gulati, Sertan Şentürk, Mohamed Sordo, Gopala Krishna Koduri, Dmitry Bogdanov, Rafael Caro, Agustín Martorell , Nadine Kroher, Ajay Srinivasamurthy, Justin Salamon, and those that I'm missing! From the MTG, I would also particularly like to thank Perfecto Herrera for his input when designing user experiments, and Alba Rosado for making me feel I could do much more than research at the MTG. 
Furthermore, I want to thank György Fazekas for his help and collaboration during my stay at Centre for Digital Music, Queen Mary University of London, and Tamsin Porter for proofreading this thesis.

Last but not least, I want to thank all my friends and family and, with a particular emphasis, I would like to thank Anna for always being there, accompanying me throughout the whole process, and letting me accompany her on her own. Thank you!


\vspace*{\fill}

\line(1,0){372}\\
\footnotesize
This thesis has been carried out at the Music Technology Group of Universitat Pompeu Fabra (UPF) in Barcelona, Spain, from October~2010 to February~2014 and from May~2014 to March~2015, and at the Centre for Digital Music of Queen Mary University of London (QMUL), United Kingdom, from March~2014 to April~2014. This work has been supported by the Spanish Ministry of Science and Innovation (BES-2010-037309 FPI grant and TIN-2009-14247-C02-01 DRIMS project), and by the European Research Council (FP7-2007-2013 / ERC grant agreement 267583). The research stay at QMUL has been also funded by the Spanish Ministry of Science and Innovation (EEBB-I-14-08838).
\normalsize