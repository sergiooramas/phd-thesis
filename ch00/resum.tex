La creació, publicació i disseminació de contingut musical ha canviat radicalment en les últimes dècades. D'una banda, grans quantitats d'informació són publicades diàriament a pàgines web, fòrums, wikis i xarxes socials. No obstant això, la major part d'aquests continguts són encara incomprensibles computacionalment, ja que són creats per i per a humans. D'altra banda, els serveis de música en línia ofereixen inesgotables catàlegs amb milions de cançons. Aquesta àmplia disponibilitat presenta dos desafiaments. Com anotar i classificar adequadament un item musical en una gran col·lecció? Com pot un usuari explorar o descobrir música del seu grat entre tot el contingut disponible? En aquesta tesi, abordem aquestes qüestions centrant-nos en l'enriquiment semàntic de descripcions d'ítems musicals (biografies d'artistes, ressenyes musicals, metadades, etc.), i en l'exploració de dades heterogenis en grans col·leccions de música (textos, àudios i imatges) . Per això, en primer lloc ens centrem en el problema d'enllaçar textos musicals amb bases de coneixement en línia, i en la construcció automatitzada de bases de coneixement musical. Després vam investigar com el coneixement extret pot impactar en sistemes de recomanació i classificació, a més de en estudis musicològics. Mostrem com el modelatge d'informació semàntica contribueix a millorar els resultats pel que fa a mètodes basats en text, tant en similitud d'artistes com en classificació de gèneres musicals, ia aconseguir millores significatives en recomanació de música pel que fa a algoritmes de referència, mentre al seu vegada es promouen recomanacions d'items menys populars. A continuació, vam investigar l'aprenentatge de noves representacions de les dades a partir de diverses modalitats de contingut usant xarxes neuronals. Seguint aquesta metodologia, emprenem el problema de recomanar nova música combinant text i àudio. Mostrem com l'enriquiment semàntic dels textos i la fusió tardana de representacions apreses millora la qualitat de les recomanacions. A més, abordem el problema de classificació de gèneres musicals amb múltiples etiquetes utilitzant text, àudio i imatges. Els experiments mostren que l'aprenentatge i la combinació de representacions de dades produeix millors resultats. Un dels fruits d'aquesta tesi és la publicació de sis datasets i dues bases de coneixement. A més, els nostres descobriments poden ser directament aplicats al disseny de nous algoritmes de recomanació de música, i més concretament, d'artistes nous i desconeguts, cosa que té potencial impacte en la indústria. Encara que la nostra investigació està motivada per les particularitats del domini de la música, creiem que les metodologies proposades poden ser fàcilment generalitzables a altres dominis.