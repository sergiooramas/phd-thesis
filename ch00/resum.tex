La creació, publicació i disseminació de contingut musical ha canviat radicalment en les últimes decades. Per una banda, una gran quantitat d’informació es publica diàriament a pàgines web, fòrums, wikis i xarxes socials. Tot i això, la majoria d’aquest contingut és encara incomprensible computacionalment degut a que es crea per i per als humans. Per una altra banda, els serveis de música online ofereixen inagotables catàlegs de milions de cançons. Aquest àmplia disponibilitat ofereix dos reptes; com es pot anotar i classificar adequadament un ítem musical en una col·lecció molt gran? I en segon lloc; com pot un usuari explorar o descobrir música del seu gust entre tot el contingut disponible? En esta tesi abordem aquestes qüestions centrant-nos el l’enriquiment semàntic de descripcions d’ítems musicals (biografies d’artistes, ressenyes musical, metadades, etc.) i en la exploració de dades heterogènies en grans col·leccions de música (textos, àudio i imatges). Ens centrem en primer lloc en el problema d’enllaçar textos musicals amb bases de coneixement online i en la construcció automatitzada d’aquestes bases de coneixement.
Tot seguit investiguem quin impacte pot tenir el coneixement extret anteriorment en sistemes de recomanació y classificació, a més de en estudis musicològics. Mostrem a continuació com el modelat de la informació semàntica contribueix a millorar els resultats obtinguts amb métodes basats només en text, tant pel que fa a la similitud d’artistes com a la classificació de gèneres musicals.  Aquest modelat també aconsegueix millores significatives en recomanació de música, en comparació a algorismes de referència, alhora que es promouen recomanacions d’ítems menys populars. A continuació investiguem l'aprenentatge de noves representacions de les dades a partir de diverses modalitats de contingut fent servir xarxes neuronals. Seguint aquesta metodologia, encarem el problema de recomanar nova música combinant text i àudio. Mostrem com l’enriquiment semàntic dels textos i la fusió tardana de representacions apreses millora la qualitat de les recomanacions. A més, abordem el problema de classificació de gèneres musicals amb múltiples etiquetes utilitzant text, àudio i imatges. Els experiments mostren que l’aprenentatge i la combinació de representacions de dades produeixen millors resultats. Un dels fruits d’aquesta tesi es la publicació de sis datasets i dues bases de coneixement. A més, els nostres descobriments es poden aplicar directament al disseny de nous algorismes de recomanació de música i, més concretament, a la recomanació d’artistes nous o desconeguts, de tal manera que té potencial per generar impacte a la indústria. Encara que la motivació d’aquesta investigació són les particularitats del domini de la música, creiem que les metodologies proposades poden ser fàcilment generalitzables a altres dominis.  
