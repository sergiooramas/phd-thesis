La creación, publicación y diseminación de contenido musical ha cambiado radicalmente en las últimas décadas. Por un lado, grandes cantidades de información son publicadas diariamente en páginas web, fórums, wikis y redes sociales. Sin embargo, la mayor parte de estos contenidos son aún incomprensibles computacionalmente, ya que son creados por y para humanos. Por otro lado, los servicios de música online ofrecen inagotables catálogos con millones de canciones. Esta disponibilidad presenta dos desafíos. Primero, ¿cómo clasificar adecuadamente un ítem musical en una gran colección? Segundo, ¿cómo puede un usuario explorar o descubrir música de su agrado entre todo el contenido disponible? En esta tesis, abordamos estas cuestiones centrándonos en el enriquecimiento semántico de descripciones de ítems musicales (biografías de artistas, reseñas musicales, metadatos, etc.), y en el aprovechamiento de datos heterogéneos presentes en grandes colecciones de música (textos, audios e imágenes). Para ello, primero nos centramos en el problema de enlazar textos musicales con bases de conocimiento online y en la construcción automatizada de bases de conocimiento musical. Luego investigamos cómo el conocimiento extraído puede impactar en sistemas de recomendación y clasificación, además de en estudios musicológicos. Mostramos cómo el modelado de información semántica contribuye a mejorar los resultados con respecto a métodos basados solo en texto, tanto en similitud de artistas como en clasificación de géneros musicales, y a conseguir mejoras significativas en recomendación de música con respecto a algoritmos de referencia, mientras a su vez se promueven recomendaciones de ítems menos populares. A continuación, investigamos el aprendizaje de nuevas representaciones de datos a partir de contenidos multimodales utilizando redes neuronales, y lo aplicamos a los problemas de recomendar música nueva y clasificar géneros musicales con múltiples etiquetas, mostrando que el enriquecimiento semántico y la combinación de representaciones aprendidas produce mejores resultados. Uno de los frutos de esta tesis es la publicación de seis datasets y dos bases de conocimiento. Además, nuestros descubrimientos pueden ser directamente aplicados al diseño de nuevos algoritmos de recomendación de música, y más concretamente, de artistas nuevos y desconocidos, lo cual tiene potencial impacto en la industria musical. Aunque nuestra investigación está motivada por las particularidades del dominio de la música, creemos que las metodologías propuestas pueden ser fácilmente generalizables a otros dominios.