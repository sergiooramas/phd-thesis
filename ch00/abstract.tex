Online sharing platforms host a vast amount of multimedia content generated by their own users. 
Such content is typically not uniformly annotated and can not be straightforwardly indexed. Therefore, making it accessible to other users poses a real challenge which is not specific of online sharing platforms. In general, content annotation is a common problem in all kinds of information systems.
In this thesis, we focus on this problem and propose methods for helping users to annotate the resources they create in a more comprehensive and uniform way. Specifically, we work with tagging systems and propose methods for recommending tags to the content creators during the annotation process.
To this end, we exploit information gathered from previous resource annotations in the same sharing platform, the so called \emph{folksonomy}.
Tag recommendation is evaluated using several methodologies, with and without the intervention of users, and in the context of large-scale tagging systems. 
We focus on the case of tag recommendation for sound sharing platforms. Besides studying the performance of several methods in this scenario, we analyse the impact of one of our proposed methods on the tagging system of a real-world and large-scale sound sharing site.
As an outcome of this thesis, one of the proposed tag recommendation methods is now being daily used by hundreds of users in this sound sharing site.
In addition, we explore a new perspective for tag recommendation which, besides taking advantage of information from the folksonomy, employs a sound-specific ontology to guide users during the annotation process.
Overall, this thesis contributes to the advancement of the state of the art in tagging systems and folksonomy-based tag recommendation, and explores interesting directions for future research. 
Even though our research is motivated by the particular challenges of sound sharing platforms and mainly carried out in that context, we believe our methodologies can be easily generalised and thus be of use to other information sharing platforms.